{\centering
\begin{longtable}{|p{2.5cm}|p{3cm}|p{8cm}|}
    \hline
    Sub architecture & Layer & Innovation capability \\ \hline
    \multirow{7}{2.5cm}{Horizontal layers} 
        & \multirow{42}{3cm}{Context} &  Proactive initiatives for identifying opportunities. \\  \cline{3-3}
        & & Procedures to manage and realise ideas. \\ \cline{3-3}
        & & Testing, screening and prioritising opportunities and ideas. \\ \cline{3-3}
        & & Ideas are quickly defined and prototyped. \\ \cline{3-3}
        & & Practices and procedures for developing and implementing ideas. \\ \cline{3-3}
        & & Practices to network and facilitate collaboration between internal teams. \\ \cline{3-3} 
        & & Procedures for identifying and exploring latent opportunities. \\ \cline{3-3}
        & & Core competencies are identified \\ \cline{3-3}
        & & Human resources are managed to ensure sufficient core competencies for operational needs. \\ \cline{3-3}
        & & Core innovation competencies are identified. \\ \cline{3-3} 
        & & Human resources are managed to ensure sufficient core competencies for research and development. \\ \cline{3-3}
        & & Procedures to ensure needed competencies are considered during the hiring process. \\ \cline{3-3} 
        & & Procedures for communication has been identified and implemented. \\ \cline{3-3} 
        & & Organisational resource needs are being monitored. \\ \cline{3-3}
        & & Sufficient resources are allocated to innovation. \\ \cline{3-3}
        & & Investment and prioritisation of innovation. \\ \cline{3-3}
        & & Organisational values and policies encourage innovation. \\ \cline{3-3}
        & & Change management procedures have been defined and deployed. \\ \cline{3-3}
        & & Initiatives for motivating, rewarding, and celebrating success. \\ \cline{3-3}
        & & Align existing personnel’s skills with their role. \\ \cline{3-3}
        & & Creating cross-functional and multidisciplinary teams. \\ \cline{3-3}
        & & Flexible organisational and human allocation structures. \\ \cline{3-3}
        & & Organisational structures that encourage organisation wide communication. \\ \cline{3-3}
        & & The organisational structure enables efficient decision-making. \\ \cline{3-3}
        & & Innovation metrics have been identified and defined. \\ \cline{3-3}
        & & Benchmarkings has been established to compare innovation metrics with successful organisations. \\ \cline{3-3}
        & & Goals are aligned with innovation objectives. \\ \cline{3-3}
        & & Innovation activities are appropriately prioritised with allocated resources.\\ \cline{3-3}
        & & Identifying and planning for important decisions. \\ \cline{3-3}
        & & Innovation process and activities are grounded in theory. \\ \cline{3-3}
        & & Innovation committee has been established or roles have been identified and assigned responsibility for key innovation related choices. \\ \cline{3-3}
        & & Identifying, documenting and implementing best-practices for innovation. \\ \cline{3-3}
        & & Identified strategy for knowledge acquisition. \\  \cline{3-3}
        & & Identified strategy for acquiring knowledge related technologies. \\  \cline{3-3}
        & & Strategy and innovation objectives are continuously improved and communicated. \\ \cline{3-3}
        & & Align project management with type(s) of innovation. \\ \cline{3-3}
        & & Innovation process competencies have been identified, acquired and developed. \\ \cline{3-3}
        & & Frameworks for contextualising, categorising and analysing data. \\ \cline{2-3}
        & \multirow{1}{3cm}{Services} & \textbf{Non} \\  \cline{2-3}
        & \multirow{1}{3cm}{Business} & \textbf{Non} \\  \cline{2-3}
        & \multirow{1}{3cm}{Application} & \textbf{Non} \\  \cline{2-3}
        & \multirow{6}{3cm}{Data Space} & Procedures for continuously understanding the needs of the end user. \\  \cline{3-3}
        & & Managing tacit knowledge. \\ \cline{3-3}
        & & Procedures for capturing, and retrieving data. \\ \cline{3-3}
        & & Practices for exploring existing and new fields of research. \\ \cline{3-3}
        & & Metrics are monitored to identify process and management improvements. \\ \cline{3-3}
        & & Procedures for identifying, summarising, highlighting, and extracting relevant information. \\ \cline{2-3}
        & \multirow{5}*{Technologies} & Core technologies are identified, managed, and maintained to ensure that project and operational needs are continuously fulfilled. \\ \cline{3-3}
        & & Procedures for proactively identifying, developing, and acquiring required technologies. \\ \cline{3-3}
        & & Tools to facilitate the information flow have been identified and implemented. \\ \cline{3-3}
        & & Tools for identifying, summarising, highlighting, and/or extracting relevant information. \\ \cline{3-3}
        & & Procedures for developing and elaborating concepts. \\  \cline{3-3}
        & & Tools and technology for storing and maintaining data. \\ \cline{2-3}
        & \multirow{1}*{Infrastructure} & Physical resources are allocated to the portfolio of projects, based on prioritisation and in balance with operational requirements.  \\ 
        \hline        
    \multirow{4}{2.5cm}{Stakeholder perspective} 
        & \multirow{3}{3cm}{Stakeholders} & Involving end user at various stages throughout the innovation process \\  \cline{3-3}
        & & Involving suppliers at various stages throughout the innovation process. \\ \cline{3-3}
        & & Involving other stakeholders (partners, alliances, etc.) in the innovation process. \\ \cline{2-3}
        & \multirow{2}{3cm}{Policies} & Procedures for ensuring supplier competency and that technology supports innovation type(s).\\  \cline{3-3}
        & & Practices to communicate and collaborate with external parties. \\ \cline{2-3}
        & \multirow{1}{3cm}{Privacy and Trust} & \textbf{Non} \\  \cline{2-3}
        & \multirow{1}{3cm}{Ownership} & Planning and coordinating the innovation portfolio. \\
         
        \hline        
    \multirow{3}{2.5cm}{Data Perspective} 
        & \multirow{1}{3cm}{Interoperability} & Opportunities and concepts are aligned and with required technology, competencies, processes, systems, etc.  \\  \cline{2-3}
        & \multirow{1}{3cm}{Data security, Risk} & Procedures to reduce project uncertainty and identify, manage, and mitigate risk. \\  \cline{2-3}
        & \multirow{3}{3cm}{Data Governance} & Managing and balancing the innovation portfolio. \\  \cline{3-3}
        & & Managing intellectual property. \\ \cline{3-3}
        & & Establish intellectual property management and sharing policy. \\ 
        \hline
    \multirow{6}{2.5cm}{Development process} 
        & \multirow{1}{3cm}{Identify component} & Opportunities and ideas are coordinated and viewed in context with required technology, competencies, processes, systems, etc.  \\  \cline{2-3}
        & \multirow{1}{4cm}{Identify relationships} & \textbf{Non} \\  \cline{2-3}
        & \multirow{1}{4cm}{Identify stakeholder} & \textbf{Non} \\  \cline{2-3}
        & \multirow{1}{3cm}{Validation} & \textbf{Non} \\  \cline{2-3}
        & \multirow{1}{3cm}{Iteration} & \textbf{Non} \\  \cline{2-3}
        & \multirow{1}{3cm}{Identify views} & \textbf{Non} \\
        \hline
    
    \caption{A mapping of the innovation capabilities discussed in \cite{louw2017architecting} table 1, to the +CityExchange EAF with changes to fit smart city development}
    \label{tab:4-innovation-capabilities}
\end{longtable}}