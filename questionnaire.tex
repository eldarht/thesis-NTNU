
Enterprise architecture: its role in +CityxChange
Request for participation to provide feedback on the +CityxChange Enterprise Architecture Framework

This is a question to you about participation in in a research project. Here we give you information about the aim of the project and what participation means for you.

Aim
The aim of this questionnaire is to obtain feedback on the Enterprise Architecture Framework (EAF) used in +CityxChange (+CxC). The questionnaire has three main parts: (i) Demographic information on respondents; (ii) feedback on the usefulness of the EAF and use case models; and (iii) how the EAF could be enhanced to support knowledge transfer within and across cities.

This work is conducted as a part of a Master’s project at the Department of Computer Science, at NTNU. The feedback on the +CxC EAF may be included as a part of the deliverable D1.2 Report on the Architecture for the ICT Ecosystem for the +CxC project.

The respondents to this questionnaire should have seen a presentation or used the +CxC EAF for modelling use cases.

Who is responsible for the research project?
NTNU IDI Is responsible for the data processing in this project.

Why are you being asked to participate?
You are associated with the +CityxChange project and have been exposed to the +CxC EAF

Prospective respondents were purposely selected and invited to partake in the survey since they have prior knowledge on enterprise architecture or/and are familiar with the developed EAF used in +CxC project. Accordingly, the email address of the selected respondents were gotten either from the +CxC project master list or from the respondents organisational website.

What does it mean for you to participate?
Participation is through an electronic questionnaire. The questions are primarily about your opinion of how Enterprise Architecture relates to your work and the usefulness of the +CxC EAF. The questions are a combination of multiple choice, likert-scale and free-text.

Participation is optional
It is optional to participate in the project. If you decide to participate, then you can opt out at any point and withdraw your consent without giving any reason. All your personal data will then be deleted. There are no negative consequences for you if you do not wish to participate or opt out later.

Your privacy – How we use or process your data

We will only use the data for purposes explained here. We process your data confidentially and in line with regulations.
    • Those that will have access to the data are: The student working on the thesis, the supervisor and the co-supervisor.    
    • The data will only be accessible to those mentioned above, deleted once the project completes and any published research  will anonymize the data.
    • The questionnaire is conducted with nettskjema. You can get more information on that here: https://www.uio.no/tjenester/it/adm-app/nettskjema/mer-om/

Any research publication will not give personally identifiable information

Your rights
As long as you can be identified by the data, you will have the right to:
    • Insight into what data we collect about you and retrieve said data.
    • Correct the data about you.
    • Delete the data about you
    • Complain about the use of data to "Datatilsynet"

What gives us the right to process personal data about you?
We process data based on your consent.

On request from NTNU IDI, NSD – Norsk senter for forskningsdata AS evaluated the processing of personal data in this project grounded in regulations.
 

How can i learn more?
If you have questions in regards to the studies, or wish to exercise your rights, contact:
    • NTNU IDI sobah.a.petersen@ntnu.no.
    • Data protection office at NTNU: Thomas Helgesen, thomas.helgesen@ntnu.no

If you have questions to NSD's evaulation of the project; contact:
    • NSD – Norsk senter for forskningsdata AS via email (personverntjenester@nsd.no) or phone: 55 58 21 17.


Regards,


Sobah Abbas Petersen, PhD
Associate Professor
Dept. of Computer Science
Norwegian University of Science and Technology
Trondheim, Norway.
Mobile: +47 92846595
Skype: Sobah1

Consent for participation in the study: I have received and understood information about the project to provide feedback on the +CityxChange Enterprise Architecture Framework.
Du må velge minst ett svaralternativ.
I give consent
Demographic Information

To understand where the feedback is coming from, we need to understand your position in your organisation and your familiarity with the Enterprise Architecture approach.

Gender?
Male

Female

Other gender identity

Prefer not to answer
Age?
<20 years

20 - 30 years

31 - 40 years

41 - 50 years

51 - 60 years

Over 61
What type of organisation do you represent?
University

Research organisation

City council or municipality

Private organisation

Public organisation

Other
If you answered other in the question above, please specify here.
What type of services does your organisation primarily provide?
Energy related

Data related

Innovation related

ICT Infrastructure related

Transport/mobility related

Other
If your answer to the question above was "Other", please indicate the type(s) of service(s) provided by your organisation.
What is your primary role within your organisation?
How much experience do you have with Enterprise Architecture
No experience

Less than 1 year

1 - 3 years

4 - 5 years

6 or more years
How much experience do you have with Smart City related projects?
No experience

Less than 1 year

1 - 3 years

4 - 5 years

6 or more years
Enterprise Architecture Approach
Indicate your level of agreement with the following statements about EA in general
 
Strongly disagree
Disagree
Neither agree nor disagree
Agree
Strongly agree
Not applicable
Enterprise architecture is relevant for my work.
Enterprise architecture is relevant for the +CxC project.
Are you familiar with the +CxC Enterprise Architecture Framework (+CxC EAF)?
I have seen a presenation of it

I have used it

I have provided feedback on the EAF

I have provided input and/or feedback to one or more models based on the EAF

I am not familiar with it

Other
Indicate your level of agreement with the following statements about the +CxC EAF
 
Strongly disagree
Disagree
Neither agree nor disagree
Agree
Strongly agree
Not applicable
The framework is useful for my work.
The framework is useful for CxC.
The framework is easy to understand.
The framework is easy to use.
I will recommend the framework to colleagues in my organisation.
I will use the framework for my work in the future.
Use Case Scenarios described using the +CxC EAF
Indicate your level of agreement with the following statements about the use case scenario models described using the +CxC EAF
 
Strongly disagree
Disagree
Neither agree nor disagree
Agree
Strongly agree
Not applicable
The use case models are useful for my work.
The use case models are useful for the +CxC project
The use case models are easy to understand
I find it easy to describe a scenario using the use case models
The use case models have helped me clarify details about our use case.
I will use the use case models for my work in the future
I will recommend the use case models to colleagues in my organisation
Enterprise architecture and knowledge transfer
Indicate your level of agreement with the following statements about how the +CxC EAF could help
 
Strongly disagree
Disagree
Neither agree nor disagree
Agree
Strongly agree
Not applicable
It could help in discussions with colleagues and/or collaboration partners within my organisation.
It could help when explaining use cases and solution architectures to colleagues.
It could help with capturing knowledge.
It could help with sharing knowledge within my organisation and/or project partners.
It could help when sharing knowledge across cities.
It could help with reusing knowledge.
Indicate your level of agreement on if +CxC EAF could support various types of activities
 
Strongly disagree
Disagree
Neither agree nor disagree
Agree
Strongly agree
Not applicable
It could support participatory design activities.
It could support collaborative activities.
It could support reflection about use cases.
It could support identifying potential value added services.
It could support creative activities such as brainstorming
It could support shared understanding to support decision making.
Are there additional information that you would like to capture using the +CxC EAF?
What techniques does your organisation use to document knowledge that individuals have learnt during a project? e.g. interviews, observations or writing documentation.
What techniques does your organisation use to share knowledge? e.g. collaboration, training or meetings.
Apart from the +CxC EAF, does your organisation use Enterprise Architecture for other means? If so, how does it relate to this framework and can they be combined?
Which problems do you think enterprise architecture can or should attempt to solve?
If you have any other feedback or comments, please add them here.
Se nylige endringer i Nettskjema (v1039_0rc261)
Vilkår Personvern og vilkår for bruk Nettskjema bruker informasjonskapsler Tilgjengelighetserklæring
Kontaktinformasjon Kontaktpunkter Nettskjema
Ansvarlig for denne tjenesten Webseksjonen – USIT
