\chapter{Methodology}
\label{chap:methodology}

This section covers the approach taken to answer the research questions and the different research methods used in this work to ensure quality.
\section{Research methodology}
\subsection{Research flow}
\begin{figure}
    \centering
    \rotatebox{-90}{
        %\missingfigure{Make diagram show research flow}
% Define block styles
\tikzset{
    objective/.style={rectangle, draw, fill=blue!20, rounded corners, text width=\textheight/4.5},
    method/.style={rectangle, draw, fill=green!20, text width=\textwidth/4.5},
    action/.style={rectangle, draw, fill=blue!20, text width=\textwidth/4.5},
    gap/.style={rectangle, draw, fill=red!20, rounded corners, text width=\textwidth/4.5},
    section/.style={rectangle, draw, fill=yellow!20, text width=\textwidth/4.5},
    line/.style={draw, -latex},
    box/.style={rectangle, draw, fill=white, align=center, text height=0.9\textwidth}
}

    
\begin{tikzpicture}[]
    
    % research gaps and problems
    \node[gap] at (-0.5,4) (citylearning) {Limited literature on how to learn from smart city projects and initiatives};
    \node[gap, below=1cm of citylearning] (ealearning) {Limited literature on how useful \gls{ea} is for learning};
    \node[gap, below=3.5cm of ealearning] (eaf+city) {The \gls{eaf} proposed in +CityxChange does not focus on learning};
    
    % Research objectives
    \node[objective] at (5.2,4) (ro1) {RO1: Gain a better understanding of how the current state of \gls{ea} facilitates learning and how the proposed smart city \gls{eaf} diverge.};
    \node[objective, below=0.3cm of ro1] (ro2) {RO2: Understand which aspects of \gls{ea} is perceived to be of use and enhance learning in smart city projects.};
    \node[objective, below=0.3cm of ro2] (ro3) {RO3:Understand how \gls{ea} can transfer and retain knowledge within an organisation and shared with other organisations.};
    \node[objective, below=0.3cm of ro3] (ro4) {RO4:Provide recommendations for improvement to the \gls{eaf} used in +CityxChange.};
    
    % Research methods
    \node[method] at (10,5.5) (literaturereview) {Literature review};
    \node[method, below= 4cm of literaturereview] (survey) {Survey method};
    \node[method, below= 3cm of survey] (model) {Model proposition};
    \node[method, below= 0.5 of model] (expert) {Expert evaluation};

    %Research actions
    \node[action, below= 0.5cm of literaturereview] (initliterature) {Initial literature review};
    \node[action, below= 0.5cm of initliterature] (literature) {Continued literature review};
    \node[action, below= 0.5cm of survey] (questionnaire) {Questionnaire with individuals working on +CityxChange};
    \node[action, below=0.5 of expert] (interview) {Interview};
    
    %Documented by
    % \node[section] at (15.3,3.7) (secliterature) {Chap1: Introduction};
    \node[section] at (15.3,3.7) (secliterature) {Chap2: Literature review};
    % \node[section, below= 2.8cm of secliterature] (secmethod) {Chap3: Methodology};
    \node[section, below= 1.5cm of secliterature] (secgathering) {Chap4: Gathering data from +CityxChange};
    % \node[section, below= 2.8cm of secliterature] (seccxcevaluation) {Chap5: +CityxChange EAF evaluation};
    \node[section, below= 2.5cm of secgathering] (secmodel) {Chap6: Proposed model};
    \node[section, below= 0.3cm of secmodel] (secevaluation) {Chap7: Model evaluation};
    \node[section, below= 0.2cm of secevaluation] (secresult) {Chap8: Results and discussion};
    % \node[section, below= 2.8cm of secliterature] (secmodel) {Chap6: Proposed model};
    
    % Draw connections);
    \path [line] (initliterature.10) -|  (12,6.5) -- node [near end, above] (informliterature) {Informs} (-2.7,6.5) |- (citylearning.west);
    \path [line] (initliterature.10) -|  (12,6.5) -- (-2.7,6.5) |- (ealearning.west);
    
    \path [line] (citylearning.east) -- node [midway, above, sloped] {led to} (ro1.west);
    \path [line] (citylearning.east) -- (ro2.west);
    \path [line] (ealearning.east) -- (ro1.west);
    \path [line] (ealearning.east) -- (ro2.west);
    \path [line] (ealearning.east) --  node [midway, above] {led to} (ro3.west);
    \path [line] (ealearning.east) -- (ro3.west);
    \path [line] (eaf+city.east) -- node [midway, above, sloped] {led to} (ro4.west);
    
    \path [line] (ro1.east) -- (initliterature.west);
    \path [line] (ro1.east) -- (literature.west);
    \path [line] (ro2.east) -- (literature.west);
    \path [line] (ro2.east) -- (survey.west);
    \path [line] (ro3.east) -- (literature.west);
    \path [line] (ro3.east) -- (survey.west);
    \path [line] (ro4.east) -- (model.west);
    
    \path[line] (survey.south) -- node [midway, right] {Using} (questionnaire.north);
    \path[line] (literaturereview.south) -- node [midway, right] {With} (initliterature.north);
    \path[line] (initliterature.south) -- node [midway, right] {Followed by} (literature.north);
    \path[line] (initliterature.east) -- node [midway, above] {reported in} (secliterature.west);
    \path[line] (literature.east) -- (secliterature.west);
    \path[line] (survey.east) -- node [midway, above] {reported in} (secgathering.west);
    \path[line] (model.south) -- node [midway, right] {Evaluated with} (expert.north);
    \path[line] (model.east) -- node [midway, above, sloped] {reported in} (secmodel.west);
    \path[line] (expert.south) -- node [midway, right] {Using} (interview.north);
    \path[line] (expert.east) -- node [midway, above, sloped] {reported in} (secevaluation.west);
    
    \path[line] (secliterature.east) -- (17.3,3.7) |- (secresult.east);
    \path[line] (secgathering.east) -- (17.3,0.9) |- (secresult.east);
    \path[line] (secmodel.east) -- (17.3,-2.9) |- (secresult.east);
    \path[line] (secevaluation.east) -- (17.3,-4.2) |- (secresult.east);
    
    % Lanes
    \begin{scope}[on background layer]
        \node[box, text width=\textheight/4.5, 
            label={Gaps in the literature}]  (existingliterature) {};
        \node[box, right=1mm of existingliterature, text width=\textheight/4.2, 
            label={Research objectives}] (objectives) {};
        \node[box, right=1mm of objectives, text width=\textheight/4.9,
            label={Research process}]  (methods) {};
        \node[box, right=1mm of methods, text width=\textheight/5,
        label={Thesis structure}]  (documented) {};
    \end{scope}
    
    % Diagram explanation
    \node[gap, above= of existingliterature] (explaingap) {<Gap in the literature/problem>};
    \node[objective, right=2mm of explaingap] (explainobjective) {<Research objective>};
    \node[method, right=2mm of explainobjective] (explainmethod) {<Research method>};
    \node[action, right=2mm of explainmethod] (explainaction) {<Research activity>};
    \node[section, right=2mm of explainaction] (explainsection){<Thesis section/chapter>};
    
\end{tikzpicture}

    }
    \caption{How research was conducted and documented in this thesis.}
    \label{fig:3-research-flow}
\end{figure}

Figure \ref{fig:3-research-flow} shows a visualisation of how the identified problems in the literature motivated the research objectives, how the objectives were reached through the research process and how the thesis documents the process and results.

The research started with an initial literature review in order to identify gaps or problems in the literature. The results of the review informed the research objectives. 
The literature review was continued after the objectives were identified and progressed alongside with the other research activities. The other research activities that were done was a survey, model proposition and expert evaluation. The survey was conducted to better understand the use of \gls{ea} in a smart city project context, how the +CityxChange \gls{eaf} in particular contributed to the project and how it and \gls{ea} in general related to +CityxChange's learning efforts.
The survey was implemented using an online questionnaire.
The data gathered from the survey was used with the data from the literature review in order to identify the requirements and to propose a model consisting of an \gls{eaf}, a development process and \gls{ea} elements. An \gls{ea} element is considered as a visual representation of an uninstantiated entity that can be used in an \gls{ea} model. When the element is instantiated it becomes a component. 
The model was evaluated with an expert evaluation, using semi-structured interviews. 

\subsection{Literature Review approach}
\begin{table}
    \centering
    \begin{tabular}{|c|p{0.8\textwidth}|}
        \hline
        ID & Question \\ \hline
        Q01 & Is the research aim clearly stated? \\ \hline
        Q02 & Is the research method clearly stated?\\ \hline
        Q03 & Is the research context clear? \\ \hline
        Q04 & Is the research grounded in theory?\\ \hline
        Q05 & are the results clearly presented?\\ \hline
        Q06 & Is validity of research discussed?\\ \hline
        Q07 & Does it discuss use of Enterprise architecture?\\ \hline
        Q08 & Does it discuss knowledge management, innovation, knowledge transfer or learning without artificial intelligence or machine learning?\\ \hline
        Q09 & Does it discuss smart city, smart city services or construction planning?\\ \hline
        Q10 & Uses a technology acceptance model or similar.\\ \hline
    \end{tabular}
    \caption{Literature screening questions and relevance questions.}
    \label{tab:sceeningQuestions}
\end{table}
\iffalse
\begin{figure}
    \centering
    \iffalse
% Define block styles
\tikzset{
    source/.style={rectangle split, split parts=2, draw, fill=blue!20, rounded corners, text width=\textheight/4.5},
    identified/.style={rectangle split, split parts=2, draw, fill=blue!20, rounded corners, text width=\textheight/4.5},
    screening/.style={diamond, draw, fill=blue!20, text centered, text width = 1.5cm},
    selected/.style={rectangle split, split parts=2, draw, fill=green!20, text width=\textwidth/4.5},
    exclusion/.style={rectangle, draw, fill=red!20, rounded corners, text width=\textwidth/4.5},
    section/.style={rectangle, draw, fill=yellow!20, text width=\textwidth/4.5},
    line/.style={draw, -latex},
}

    
\begin{tikzpicture}[]
    % Queries
    \node[Source] () {
        \nodepart{one}
            Query
        \nodepart{two}
             "Knowledge transfer"  AND "Enterprise Architecture"
    }
    \node[Source] () {
        \nodepart{one}
            Query
        \nodepart{two}
            "Enterprise Architecture" AND "smart cities"
    }
    \node[Source] () {
        \nodepart{one}
            Query
        \nodepart{two}
            cities as learning innovation ecosystems
    }
    \node[Source] () {
        \nodepart{one}
            Query
        \nodepart{two}
            knowledge transfer across cities
    }
    \node[Source] () {
        \nodepart{one}
            Query
        \nodepart{two}
            Learning from "Enterprise Architecture models"
    }
    
    % Other sources
    \node[Source] () {
        \nodepart{one}
            Mentioned
        \nodepart{two}
            From author
    }
    \node[Source] () {
        \nodepart{one}
            Mentioned
        \nodepart{two}
            From reference
    }
    \node[Source] () {
        \nodepart{one}
            Mentioned
        \nodepart{two}
            From supervisor
    }
    
    % Identified
    \node[identified] () {
        \begin{itemize}[leftmargin=0.3cm]
            \item total selected (n=89)
        \end{itemize}
    }
    
    % Screening
    \node[screening] () {Screening}
    
    %Exclusion 
    \node[exclusion, right of screening] {Rejected (n=18)}
    
    %Pending
    
\end{tikzpicture}
\fi
    \caption{Inclusion and exclusion of literature}
    \label{fig:screening}
\end{figure}
 \fi
\begin{itemize}
    \item \textbf{Query construction:} Literature was queried using the search engines at google scholar, web of science, Scopus and Oria. The search terms used included: "Enterprise Architecture" AND "smart cities", Learning from "Enterprise Architecture models", cities as learning innovation ecosystems, and knowledge transfer across cities. Additional literature was gathered based on references and authors.
    
    \item \textbf{Screening:} Relevance and quality was assessed to exclude articles given by the query. The questions used to assess quality and relevance are listed in table \ref{tab:sceeningQuestions}. \iffalse and the result of the query and screening is displayed in figure \ref{fig:screening}. \fi
    
\end{itemize}

\subsection{Survey approach}
\begin{itemize}
    \item \textbf{Data collection method:} A questionnaire was used to gather data. The questions were based on the initial literature review discussed in chapter \ref{chap:literature} and based on the \gls{tam} to indicate potential of the \gls{eaf} proposed in +CityxChange.
    The questionnaire was made primarily for this thesis, but also to be used for a delivery within +CityxChange project \cite{cityxchange-d1.2}. As a result not all questions of the questionnaire were deemed relevant for this thesis. The questionnaire consisted of 6 parts; information about the questionnaire and consent to use and publish data, demographic on participants, view of \gls{ea} in general, view of \gls{eaf} used in +CityxChange, view of how \gls{ea} in general and the \gls{eaf} in particular relates to knowledge transfer and a final section for free-text answers with the opportunity to give feedback that the participants felt were missing or required clarification.
    In total there were 47 questions and the questions were given via nettskjema.no due to an existing data processing agreement with \gls{ntnu}. Permission to process personal data was granted by the \gls{nsd} and all participant consented to the use of relevant data.
    \item \textbf{Population/sampling:}
    The questionnaire was given online to 42 participants with a connection to +CityxChange. The criteria for receiving the questionnaire was a familiarity with the \gls{eaf} used in +CityxChange. Of the 42 asked participants 13 participated. A question within the questionnaire asked for their familiarity with the \gls{eaf} on which one participant indicated that they were not familiar with the \gls{eaf}. Although this indicate that this person did not fit the criteria for receiving and answering the questionnaire, the answers were still included in the analysis. 
    \item \textbf{Data analysis:} Due to the limited number of answers, a qualitative approach were used for most questions including all free-text questions. Percentages and graphics were calculated and created by nettskjema. 
    \iffalse Analysis was conducted using \gls{spss}. \fi
\end{itemize}

\subsection{Model proposition approach}    
The model proposition followed a designed and creation process with five steps
\begin{itemize}
    \item \textbf{Awareness:} The needs of the model were gathered through the literature review and questionnaire.
    \item \textbf{Suggestion:} A potential solution to the needs were suggested.
    \item \textbf{Development:} A model implementing the suggestions were developed.
    \item \textbf{Evaluation:} The model was evaluated with expert evaluation
    \item \textbf{Conclusion:} The results were analysed and reported in this thesis.
\end{itemize}

\subsection{Expert evaluation approach}
{\centering
\begin{longtable}{|p{2.3cm}|p{4.5cm}|p{5cm}|}
    \hline
        Themes & Questions & Motivation\\ \hline
        \multirow{1}{2.3cm}{Role of \gls{ea}} 
            & \multirow{2}{4.5cm}{First off, I would like to know a bit about how you view \gls{ea} as a whole. So, What do you consider to be the job of \gls{ea}?} & Lead off with a question the participant is likely to already have an opinion on. \\ \cline{3-3}
            & & Understand whether or not the proposed model is aligned with the needs of the participant. \\
        \hline
        \multirow{3}{2.3cm}{Participation,  Malleability, annotation and shared syntax} 
            & \multirow{2}{4.5cm}{Who do you think has a use for the \gls{ea} models?} & Understand the usage context. \\ \cline{3-3}
            & & Understand if the \gls{eaf} is aligned with the users needs. \\ \cline{2-3}
            & \multirow{1}{4.5cm}{Do you think the suggested \gls{eaf} allow them (groups of users) to adequate model their perspective of the \gls{ea}? (Why, why not?)} & Understand if it supports the different usage contexts \\ \cline{3-3}
            & & Understand if there are perspectives with misssing elements. \\ \cline{2-3}
            & \multirow{2}{4.5cm}{Do you think that any part of the \gls{eaf} could be interpreted differently by some of the users? (Why, why not?)} & Understand if the users would have a common understanding of the resulting model. \\ \cline{3-3}
            & & Understand if the syntax or terminology has contradictory meanings within between different communities. \\
        \hline
        \multirow{2}{2.3cm}{Visualization, Accessibility} 
            & \multirow{1}{4.5cm}{Do you think the \gls{eaf} would give an efficient representation of the \gls{ea}? (why, why not)} & Understand whether or not the visualization is easy to interpret and important components can quickly be identified. \\ \cline{2-3}
            & \multirow{1}{4.5cm}{Do you think the \gls{eaf} will be understood by non-practitioners of \gls{ea}? (why, why not)} & Understand if the complexity is too high or if it creates problems when using the model to introduce the \gls{ea} to new personnel. \\
        \hline
        \multirow{2}{2.3cm}{+CityxChange, concreteness} 
            & \multirow{1}{4.5cm}{Do you think the \gls{eaf} is adequate for +CityxChange projects? (why, why not)} & Understand if the needs of +CityxChange is aligned with the \gls{eaf}. \\ \cline{2-3}
            & \multirow{1}{4.5cm}{Is there anything you think is unnecessary in the \gls{eaf}? (why, why not)} & Understand whether or not the \gls{eaf} is to broad or can be simplified without negative effects. \\
        \hline
        \multirow{1}{2.3cm}{Modularity} 
            & \multirow{2}{4.5cm}{Would the \gls{eaf} allow specific problems to be viewed separately from the entire \gls{ea} while maintaining correctness or validity? (why, why not) } & Understand if the \gls{eaf} supports modularity. \\ \cline{3-3}
            & & Understand if the different users of the model can work on what is relevant to them without unnecessary interference. \\
        \hline
        \multirow{3}{2.3cm}{Technology acceptance} 
            & \multirow{1}{4.5cm}{Do you think the \gls{eaf} would be usefull for +CityxChange or smart city development?} & Understand "perceived usefulness" as it relates to \gls{tam} and +CityxChange and the smart city development projects. \\ \cline{2-3}
            & \multirow{1}{4.5cm}{Do you think the \gls{eaf} would be easy to use?} & Understand "perceived ease of use" as it relates to \gls{tam}. \\ \cline{2-3}
            & \multirow{1}{4.5cm}{Would you use this yourself or recommend it to others?} & Understand "intent to use" as it relates to \gls{tam}. \\
    \hline
    
    \caption{Interview questions for expert evaluation of proposed EAF}
    \label{tab:4-evaluation-interview}
\end{longtable}}
\begin{itemize}
    \item \textbf{Data collection method:} A semi structured interview was used for the evaluation. The interview consisted of six themes; "Role of \gls{ea}", "Participation,  Malleability, annotation and shared syntax", "Visualization, Accessibility", "+CityxChange, concreteness", "Modularity", "Technology acceptance". The themes and relevant questions are listed in table \ref{tab:4-evaluation-interview}. The themes are based on the \gls{boundary object} discussed in \cite{abraham2015crossing} and \gls{tam}. 12 questions were prepared before the interview and used to guide the interview. 30 minutes were allocated to each interview as the participants were busy and requested that it be kept short. Not all questions were asked during every interview due to the time constraint and nature of semi structured interviews. The themes were still covered. The interviewees were given a short introduction to the proposed model consisting of the proposed \gls{eaf}, development process and \gls{ea} elements, but were not given definitions of the layers or use case for the elements, unless explicitly requested. This was done to determine if the terminology used was appropriate and intuitive. Intuitiveness was one of the problems identified through the questionnaire and seen as important for the evaluation.
    \item \textbf{Population/sampling:} Three participants were selected and interviewed separately. All three were selected based on their knowledge of either the +CityxChange \gls{eaf} or learning within smart city projects.
    The qualities of the participants that were important for the sampling process were: 
    \begin{itemize}
        \item Participant 1: Experienced with \gls{ea} and one of the architects behind the \gls{eaf} used in +CityxChange.
        \item Participant 2: Familiar with \gls{ea} for smart cities and a contributor to the \gls{eaf} used in +CityxChange.
        \item Participant 3: Not familiar with \gls{ea}, but experienced with urban planning and learning across cities. 
    \end{itemize}
\end{itemize}