\chapter{Literature review} % Do not add own contribution and include figures/tables adapted from the literature
\label{chap:literature}

\begin{figure}
    \centering
    \tikzstyle{every node}=[draw=black,thick,anchor=west]
\tikzstyle{selected}=[draw=red,fill=red!30]
\tikzstyle{optional}=[dashed,fill=gray!50]
\begin{tikzpicture}[%
  grow via three points={one child at (-0.8,-0.7) and
  two children at (-0.8,-0.7) and (-0.8,-1.4)},
  edge from parent path={(\tikzparentnode.195) |- (\tikzchildnode.west)}]
  \node {2 Literature review}
    child { node {2.1 Research context}}		
    child { node {2.2 Overview of study area} 
        child { node {Boundary object approach to learning using EA}}
        child { node {KA and KM approach to learning using EA}}
        child { node {Other approaches to learning using EA}}
    }
    child [missing] {}				
    child [missing] {}				
    child [missing] {}
    child { node {2.3 Review of current practices}}
    child { node {2.4 Related works}}
    child { node {2.5 Summary}};	
\end{tikzpicture}
    \caption{Structure of Literature review chapter}
    \label{fig:2-TOC}
\end{figure}
 
This section summarises the previous work that cover the same or a similar topic and explores potential research gaps. It also illustrates the current state of the art in smart city \gls{ea}.

The structure of the chapter is shown in figure \ref{fig:2-TOC}.
\section{Research context}

\begin{figure}
    \centering
    %\begin{turn}{-90}
    \makebox[\textwidth][c]{
        \input{figures/tikz/4-architecture-before}
    }
    %\end{turn}
    \caption{EAF used in +CityxChange adapted from \cite{cityxchange-d1.2}}
    \label{fig:4-architecture-before}
\end{figure}


This literature review was conducted to better understand the state of \gls{ea} in smart city projects or similar projects and understand the current research gaps related to \gls{ea} as a tool for learning. It was initiated as a result of work related to +CityxChange where researchers found that their \gls{eaf} might be improved by considering how enterprises learn. Their work is in part documented in \cite{cityxchange}. 
The \gls{eaf} shown in figure \ref{fig:4-architecture-before} was proposed in +CityxChange along with the development process shown in figure \ref{fig:4-architecture-development} and will be used in this thesis as a base to be improved in relation to learning. 
The \gls{eaf} was created for representing \glspl{ict ecosystem} involving multiple stakeholders. Their use of \gls{ict ecosystem} builds on \cite{ictecosystem} that describes \glspl{ict ecosystem} as "encompasses the policies, strategies, processes, information, technologies, applications and stakeholders that together make up a technology environment for a country, government or an enterprise. Most importantly, an ICT ecosystem includes people - diverse individuals who create, buy, sell, regulate, manage and use technology." \cite[p.~3]{ictecosystem}. +CityxChange builds on this description "+CityxChange encompasses not only the data, applications and technologies, but also the policies, regulations, processes, and stakeholders that together constitute the larger technology environment for implementing +CityxChange solutions in each of the cities." \cite[p.~117]{10.1007/978-3-030-22482-0_9}.

The horizontal layers in figure \ref{fig:4-architecture-before} can be referred to as the technology stack. It adds upon terminology used in \gls{togaf}. Data is considered to be an important aspect of the \gls{eaf} as many services and stakeholders rely on data in general and open data in particular. Physical infrastructure is important as smart city development projects often involve physical assets such as electrical grids or measurement devices. 
The context layer contains the drivers for the services being developed. These drivers unify the partners involved in the project. 
The vertical layers describe components that go across the horizontal layers and might be connected to several layers simultaneously. These are often values that effect the system as a whole and not individual components. 

\begin{figure}
    \centering
    
% Define block styles
\tikzset{
    decision/.style={diamond, draw, fill=blue!20, text centered, text width = 1.5cm},
    block/.style={rectangle, draw, fill=blue!20, rounded corners, text width = 1.5cm},
    line/.style={draw, -latex},
    textnode/.style={text width=0.5cm}
}

    
\begin{tikzpicture}[]
    % Place nodes
    \node [block] (components) {Identify and describe components in horizontal layers};
    \node [block, right= 0.8cm of components] (relationships) {Identify and describe relationships};
    \node [block, right= 0.8cm of relationships] (perspectives) {Identify stakeholder and data perspectives};
    \node [decision, right= 0.8cm of perspectives] (iscomplete) {Is model complete};
    \node [block, above= 0.8cm of relationships] (iterate) {Iterate to add detail};
    \node [block, right= 0.8cm of iscomplete] (views) {Identify views to visualise};

   
    % Draw edges
    \path [line] (components.east) -- (relationships.west);
    \path [line] (relationships.east) -- (perspectives.west);
    \path [line] (perspectives.east) -- (iscomplete.west);
    \path [line] (iscomplete.north) |- node [right] {no} (iterate.east);
    \path [line] (iterate.west) -| (components.north);
    \path [line] (iscomplete.east) -- node [above] {yes} (views.west);
\end{tikzpicture}


    \caption{Development process of EA models using the +CityxChange EAF, adapted from \cite{cityxchange-d1.2} figure 5.1}
    \label{fig:4-architecture-development}
\end{figure}

\cite{cityxchange} gives some guiding principles on how the \gls{eaf} could be used, but leaves out specifics so as to allow for greater flexibility for \gls{ea} architects. Figure \ref{fig:4-architecture-development} shows the proposed development process.

\begin{figure}
    \centering
    %\begin{turn}{-90}
    \tikzset{
  box/.style={
    draw,
    rectangle,
    fill=white,
    align=center,
  },
  application/.style={draw, rectangle, fill=blue!20},
  api/.style={draw, rectangle split, rectangle split parts=2, rectangle split part fill={green!20,blue!20}},
  repository/.style={draw, cylinder, rotate=90, fill=blue!20},
  connect/.style={draw, -latex},
  realisation/.style={draw, dashed, -latex}
}

\begin{tikzpicture}[]

    % Context - Inner
    \begin{scope}[on background layer]
        \node[box, text width=\textwidth, text height=\textheight/9]  (context) {};
        \node[box, below=1mm of context, text width=\textwidth, text height=\textheight/9] (service) {};
        \node[box, below=1mm of service, text width=\textwidth, text height=\textheight/9]  (business) {};
        \node[box, below=1mm of business, text width=\textwidth, text height=\textheight/9]  (application) {};
        \node[box, below=1mm of application, text width=\textwidth, text height=\textheight/9]  (+cityxchange) {};
        \node[box, below=1mm of +cityxchange, text width=\textwidth, text height=\textheight/9]  (technology) {};
        \node[box, below=1mm of technology, text width=\textwidth, text height=\textheight/9]  (physical) {};
    \end{scope}

    %Layer text
    \node[box, left=-2cm of context, text width=2cm, minimum height=\textheight/9+2.5mm] (cstart) {Context Layer};
    \node[box, left=-2cm of service, text width=2cm, minimum height=\textheight/9+2.5mm] (sstart) {Service Layer};
    \node[box, left=-2cm of business, text width=2cm, minimum height=\textheight/9+2.5mm] (bstart) {Business Layer};
    \node[box, left=-2cm of application, text width=2cm, minimum height=\textheight/9+2.5mm] (astart) {Application and Data Processing Layer};
    \node[box, left=-2cm of +cityxchange, text width=2cm, minimum height=\textheight/9+2.5mm] (dstart) {Data Space Layer};
    \node[box, left=-2cm of technology, text width=2cm, minimum height=\textheight/9+2.5mm] (tstart) {Technologies Layer};
    \node[box, left=-2cm of physical, text width=2cm, minimum height=\textheight/9+2.5mm] (pstart) {Physical Infrastructure Layer};
    
    %Context layer
    \node[box, right=2mm of cstart, text width= 3.2cm] (cindicator) {Indicator for eMaaS uptake};
    \node[box, right=2mm of cindicator, text width= 3.2cm] (ccontibute) {Contribute to uptake of eMaas};
    \node[box, right=2mm of ccontibute, text width= 3.2cm] (cseamless) {Seamless eMobility concept (of TK)};
    
    %service layer
    \node[box, right=2mm of sstart, text width= 3.2cm] (straffic) {Traffic Management};
    \node[box, right=2mm of straffic, text width= 3.2cm] (sservice) {eMaas Service};
    \node[box, right=2mm of sservice, text width= 3.2cm] (sgreen) {Green Mobility Management};
    
    %Business layer
    \node[box, right=2mm of bstart, text width= 2cm] (bpowel) {Powel};
    \node[box, below=2mm of bpowel, text width= 2cm] (babb) {ABB};
    \node[box, right=2mm of bpowel, text width= 2cm] (btk) {TK};
    \node[box, below=2mm of btk, text width= 2cm] (bfourc) {FourC};
    \node[box, right=2mm of btk, text width= 2cm] (batb) {AtB};
    \node[box, below=2mm of batb, text width= 2cm] (biota) {IOTA};
    
    %Application layer
    \node[api, right=2mm of astart] (apay){API \nodepart{two} AtB Payment ?};
    \node[application, right=2mm of apay, text width= 2.8cm] (atraffic) {Total Traffic Control {TTC} Application};
    \node[application, right=2mm of atraffic, text width= 2.8cm] (aapp) {eM App {Android\textbackslash Web}};
    \node[api, right=2mm of aapp] (aiota) {API \nodepart{two} IOTA ?};  
    
    %Data Space Layer
    \node[repository, right=8mm of dstart,  xshift=5mm, label=left:{AtB}] (datb) {};
    \node[repository, below=12mm of datb, label=left:{Flight info}] (dflight) {};
        \node[box, draw, fill=green!20,xshift=0.5cm,yshift=0.1cm] at (dflight.south) {api};
    \node[repository, right=25mm of dstart, xshift=5mm, label=left:{Nabobil}] (dnabobil) {};
    \node[repository, below=12mm of dnabobil, label=left:{ENTUR}] (dentur) {};
        \node[box, draw, fill=green!20,xshift=0.5cm, yshift=0.1cm] at (dentur.south) {api};
    \node[repository, right=42mm of dstart, xshift=5mm, label=left:{Road Datex}] (ddatex) {};
    \node[repository, below=12mm of ddatex, label=left:{ABB DB}] (dabb) {};
        \node[box, draw, fill=green!20,xshift=0.5cm, yshift=0.1cm] at (dabb.south) {api};
    \node[repository, right=59mm of dstart, xshift=5mm, label=left:{TTC}] (dttc) {};
        \node[box, draw, fill=green!20,xshift=0.5cm, yshift=0.1cm] at (dttc.south) {api};
    \node[repository, below=12mm of dttc, label=left:{City Bikes}] (dbikes) {};
        \node[box, draw, fill=green!20,xshift=0.5cm, yshift=0.1cm] at (dbikes.south) {api};
    \node[repository, right=78mm of dstart, xshift=5mm, label=left:{AVIS}] (davis) {};
        \node[box, draw, fill=green!20,xshift=0.5cm, yshift=0.1cm] at (davis.south) {api};
    \node[repository, below=12mm of davis, label=left:{Taxi}] (dtaxi) {};
    \node[repository, right=100mm of dstart, xshift=5mm, label=left:{IOTA}] (diota) {};
    
    
    %Technology Layer
    \node[box, right=2mm of tstart, text width= 2cm] (tradar) {Radars{\color{red}?}};
    \node[box, right=6.4cm of tradar, text width= 2cm] (tiota) {IOTA};
    
    %Physical Infrastructure Layer
    \node[box, right=2mm of pstart, text width= 1.1cm] (pflights) {Flights};
    \node[box, right=2mm of pflights, text width= 1.1cm] (pbuses) {AtB Buses};
    \node[box, right=2mm of pbuses, text width= 1.1cm] (pev) {EVs};
    \node[box, right=2mm of pev, text width= 1.5cm] (pcharging) {EV Charging station};
    \node[box, right=2mm of pcharging, text width= 1.1cm] (pbikes) {City bikes};
    \node[box, right=2mm of pbikes, text width= 1.1cm] (ptaxis) {ptaxis};
    
    %connections from Physical
    \path [connect] (pbikes.north) -- (dbikes.west); %  (2.3,-14.5)
    \path [connect] (pbuses.north) -- (dentur.west);
    \path [realisation] (pev.north) -- (dabb.west);
    \path [connect] (pcharging.north) --(dabb.west);
    \path [connect] (pflights.north) -- (tradar.south);
    \path [connect] (ptaxis.north) -- (dtaxi.west);
    
    %connections from Technologies
    \path [connect] (tradar.north) -- node[right, near start]{{\color{red}?}} (dflight.west);
    \path [connect] (tiota.north) -- (diota.west);
    
    %Connections from Data space layer
    \path [connect] (dnabobil.east) -- (atraffic.south);
    \path [connect] (ddatex.east) -- (atraffic.south);
    \path [connect] (dentur.east) -- (atraffic.south);
    \path [connect] (datb.east) -- node[right, near start]{{\color{red}?}} (apay.south);
    \path [connect] (davis.east) -- (atraffic.south);
    \path [connect] (davis.east) -- (aiota.south);
    \path [connect] (dabb.east) -- (atraffic.south);
    \path [connect] (dbikes.east) -- (atraffic.south);
    \path [connect] (dttc.east) -- (atraffic.south);
    \path [connect] (dflight.east) -- (atraffic.south);
    \path [connect] (dtaxi.east) -- (atraffic.south);
    \path [connect] (diota.east) -- (aiota.south);
    %node[box, draw, fill=green!20,right, near start]{api}
    %Connections from Application layer
    \path [connect] (apay.north) |- node[right, near start]{{\color{red}?}} (2.5,-7) -| (aapp.130);
    \path [connect] (atraffic.60) -- (0.4, -4.5) -| (straffic.south);
    \path [connect] (aapp.north) -- (3.2,-4.5) -| (sservice.south);
    \path [connect] (aiota.north) |- (4.5,-7) -| (aapp.50);
    
    %Connections from Business layer
    
    %Connections from Service layer
    \path [connect] (sgreen.north) -- (cseamless.south);
    
    %Connections from context layer
\end{tikzpicture}

    %\end{turn}
    \caption{Example of EA model created using the +CityxChange architecture, adapted from figure 5.4 in \cite{cityxchange-d1.2}}
    \label{fig:4-architecture-example}
\end{figure}

Figure \ref{fig:4-architecture-example} shows an example \gls{ict ecosystem} or \gls{ea} model made using the +CityxChange \gls{eaf}.
It only shows the horizontal layers of the system being developed and not the stakeholder perspective or data perspective. The \gls{ea} relates to an \gls{emaas} system. It captures a multi stakeholder project with six identified partners involved in development. These are shown in the business layer. It also shows that the services rely heavily on physical infrastructures and data. 
Although the +CityxChange \gls{eaf} in figure \ref{fig:4-architecture-before}, the development process in figure \ref{fig:4-architecture-development} and the resulting \gls{ea} model in figure \ref{fig:4-architecture-example} were not evaluated on learning, it was developed based on literature on \gls{ea} and smart cities and is believed to cover important aspects of smart city development well. 

\section{Overview of study area}
The field of \gls{ea} has matured since the arrival of the Zachman framework \cite{zachman1987framework} widely regarded as the origin of \gls{ea} as a concept. There exist models to evaluate and compare \gls{ea} \cite{10.1007/978-3-642-24511-4_13} as well as a comprehensive industry for creating and maintaining \gls{ea} in organisations \cite{cameron2013analyzing}. 
Learning within organisations has also been covered in research, but not with unified concepts. 

\subsection{Boundary object approach to learning using EA}
In \cite{doi:10.1177/0162243910377624} \glspl{boundary object} are described as "[\glspl{boundary object}] form the boundaries between groups through flexibility and shared structure—they are the stuff of action" \cite[p.~603]{doi:10.1177/0162243910377624} where boundaries refer to shared spaces or objects. The article mentions that the concept was originally made to analyse cooperative work. The \glspl{boundary object} is where communication happens between group or the method used for communication. The article end with explaining that as the \glspl{boundary object} are meant for simplifying analysis, their definition is tied to scope and scale. \glspl{boundary object} are tied to scope as they must be relevant for the context that is being analysed, and they are tied to scale as the objects must be important enough to warrent analysis.
\cite{abraham2015crossing} looked at \gls{ea} models for learning using \gls{boundary object} perspective. \glspl{boundary object} might be documentation such as \gls{ea} that contain information and can be interpreted differently by individuals based on their background or occupation within an organisation.
\begin{figure}
    \centering

    
% Define block styles
\tikzset{
    property/.style={rectangle, draw, fill=blue!20, text centered, text width = 3cm},
    capacity/.style={rectangle, draw, fill=blue!20, rounded corners, text width = 4cm},
    line/.style={draw, -triangle 90},
    textnode/.style={text width=0.5cm},
}

    
\begin{tikzpicture}[]

    \node [property] (participation) {Participation};
    \node [property, above= 0.5cm of participation] (malleability) {Malleability};
    \node [property, below= 0.5cm of participation] (uptodateness) {Up-to-dateness};
    
    \node [capacity, right= 2cm of participation] (pragmatic) {Pragmatic capacity};
    \node [capacity, below= 2cm of pragmatic] (semantic) {Semantic capacity};
    
    \node [property, left= 2cm of semantic] (annotation) {Annotation};
    \node [property, below= 0.5cm of annotation] (visualization) {Visualization};
    
    \node [capacity, below= 3cm of semantic] (syntactic) {Syntactic capacity};
    
    \node [property, left= 2cm of syntactic] (concreteness) {Concreteness};
    \node [property, above= 0.5cm of concreteness] (accessibility) {Accessibility};
    \node [property, below= 0.5cm of concreteness] (modularity) {Modularity};
    \node [property, below= 0.5cm of modularity] (sharedsyntax) {Shared syntax};

    % Draw edges
    \path [line] (participation.east) -- (pragmatic.west);
    \path [line] (malleability.east) -- (pragmatic.north west);
    \path [line, loosely dashed] (uptodateness.east) -- (pragmatic.south west);
    
    \path [line] (annotation.east) -- (semantic.west);
    \path [line] (visualization.east) -- (semantic.south west);
    
    \path [line] (concreteness.east) -- (syntactic.west);
    \path [line, loosely dashed] (accessibility.east) -- (syntactic.north west);
    \path [line] (modularity.east) -- (syntactic.west);
    \path [line] (sharedsyntax.east) -- (syntactic.south west);
    
    \path[line] (semantic.north) -- (pragmatic.south);
    \path[line] (syntactic.north) -- (semantic.south);

\end{tikzpicture}



    \caption{Knowledge boundary properties and how they affect capacity. Adapted from \cite{abraham2015crossing}}
    \label{fig:knowledge-boundary-properties}
\end{figure}

The research aims to find the properties of \gls{ea} models that enable syntactic, semantic or pragmatic capacity for \glspl{boundary object}. knowledge is transferred between groups or individuals using \glspl{boundary object} and knowledge is translated. Its literature review found 11 boundary object properties;  modularity, Abstraction, concreteness, shared syntax, malleability, visualization, annotation, versioning, accessibility, up-to-dateness, stability and participation. They hypothesised that accessibility, concreteness, modularity and shared syntax increase syntactic capacity of \glspl{boundary object}. while annotation and visualization increase semantic capacity and malleability, participation and up-to-dateness increase pragmatic capacity. syntactic capacity increases semantic capacity which in turn increase pragmatic capacity. Their theory is that for the ability to learn one needs the capacity of \glspl{boundary object} and capabilities. Figure \ref{fig:knowledge-boundary-properties} shows the relationship between properties and capacities. The findings did not support the hypothesis of causation from availability and up-to-dateness to syntactic capacity, but postulate  that it is a requirement for learning. The conclusion is that \glspl{boundary object} should be connected to the domain concretely and that the visualisation should be efficient to enhance learning. 
\begin{table}
    \centering
    
    \begin{tabular}{|l|p{\textwidth/2}|}
        \hline
        Property & short Explanation \\ \hline
        Malleability & Supports changes by all communities using the \gls{boundary object}.  \\ \hline
        Participation & The relevant communities participate in the creation and maintenance of the \gls{boundary object}.  \\ \hline
        Up-to-dateness & The \gls{boundary object} is updated and communities are informed. \\ \hline
        Annotation & Individual communities can add additional information for local use.  \\ \hline
        Visualization & The \gls{boundary object} has a physical representation. \\ \hline
        Accessibility & The \gls{boundary object} is known about and accessible to the communities. \\ \hline
        Concreteness & The \gls{boundary object} contains information relevant for the specific communities. \\ \hline
        Modularity & Parts of the \gls{boundary object} can be viewed in seclusion from the rest while maintaining correctness. \\ \hline
        Shared syntax & A common understanding exist for interpretation of the \gls{boundary object}.  \\ 
        \hline
    \end{tabular}
    \caption{A short explanation of Boundary object properties from \cite{abraham2015crossing}}
    \label{tab:2-boundary-properties}
\end{table}




Table \ref{tab:2-boundary-properties} shows a short explanation of the properties.

\subsection{KA and KM approach to learning using EA}
\cite{10.1371/journal.pone.0127005} looks at \gls{ea} through the lens of \gls{ka} and \gls{km} specifically within large scale organisations. They note that \gls{ea} changed from a classic perspective, focusing on domain specific systems to large-scale architecting with focus on abstract, meta-level systems with more intensive communication infrastructures. This shift required more complex architectures. \gls{ka} is formed by knowledge reservoirs and knowledge flows and is seen as a component of enterprise assets similarly to \glspl{boundary object}. The research views \gls{ka} as "incorporates the manner of creating knowledge, its application and learning within enterprises."\cite[p.~4]{10.1371/journal.pone.0127005} The elements of \gls{ka} are people, processes, behaviours, technology and content. They conducted a literature review on \gls{ka} and found that \gls{ka} did not sufficiently address large-scale architecting, did not have suitable methodology and did not have a supervising framework. The research proposes a \gls{ka} methodology and framework to alleviate these problems. They base their \gls{ka} framework on zachman's \gls{eaf} as it is seen as an accepted standard that is both malleable formal and robust. In their framework the focus is on the planner perspective, owner perspective and designer perspective, while the other perspectives are seem as outside the scope of \gls{ka}. Their methodology is based on CommonKADS, a methodology commonly used for engineering in \gls{km} where the goal of \gls{km} is to create  models for knowledge recounting that can either be in the context category, concept category or artefact category. The researchers used "leadership, culture and structure, processes, explicit knowledge, implicit knowledge, knowledge hubs and centers, market leverage, measures, personnel skills and technological infrastructure"\cite[p.~17]{10.1371/journal.pone.0127005} as the metrics to evaluate their framework.

\subsection{Other approaches to learning using EA}
\cite{narman2016using} conducted a literature review on theoretical approaches for creating and evaluating organisational structures impact on motivation and learning. They found that most approaches were insufficient for an evaluation framework and advocated for theories with a holistic approach to organisation modelling. They selected Mintzberg \cite{mintzberg1979structuring} for their research. They used it with \gls{uml} and \gls{ocl} to create an evaluation model. 

\section{Review of current practices}
The use of \gls{ict} architecture and \gls{ea} in smart cities varies greatly. There is currently no best practices for determining what \gls{ea} to use or \gls{ict} architecture patterns to use.
\cite{kakarontzas2014} looked at important properties of smart cities that would be architecturally significant and important for deciding \gls{ict} infrastructure. It also looked at the current business aspects of the \gls{it} support infrastructure. It conducted a questionnaire comprising of questions regarding architecture, data sources, management, funding and project objectives. It found that organisational structure, business processes, information systems and infrastructure were the most important dimensions for \gls{ea}. The research conclude that the \gls{ict} architecture should be generic with a focus on interoperability and that performance was not a critical concern. They suggest the \gls{ict} architecture to use a layered architecture and \gls{mvc} pattern with \gls{api} facade and messaging architecture.
However \cite{7580810} concluded from their research that no \gls{ict} architecture would be generalizable enough to benefit new smart city projects. They ascertain that \gls{adm} is a good approach to smart city development and that smart cities can be viewed as enterprises. This is supported by \cite{pourzolfaghar2016types} which focused on the business aspect of \gls{ea}. They found that the abstract architectures proposed did not fulfil the business requirements and also recommend \gls{adm}. \cite{7580810} separated the \gls{adm} into three parts; Why, what and how, then looked at how the literature related to those separations. \gls{adm} was found to sufficiently cover the smart city issues in the literature. The issues discussed in the paper did not cover learning or knowledge transfer, so it is uncertain if \gls{adm} would be sufficient when focused on learning.

\section{Related work}
\setlength\LTleft{-3.3cm}
\setlength\LTright{+3.3cm}
\begin{longtable}{
    |p{\textheight/7}|p{\textheight/8}|p{\textheight/10}|p{\textheight/8}|
     p{\textheight/3}|
}
    \hline
    Authors & article & Purpose & context and categorisation & Model \\ \hline
        
    Kakarontzas, George - Anthopoulos, Leonidas - Chatzakou, Despoina -Vakali, Athena
    & A Conceptual Enterprise Architecture Framework for Smart Cities - A Survey Based Approach 
    & Propose generic \gls{ict} architecture 
    & \begin{itemize}[leftmargin=0.3cm]
        \item Context: EADIC - (Developing an Enterprise Architecture for Digital Cities)
        \item Categories: \gls{ict} architecture and Smart Cities
    \end{itemize} 
    & ICT architecture: host organisation of an application has a \gls{ui} \gls{mvc} layer with synchronous \gls{api} calls to Business logic layer that communicates with local data storage and \gls{mom} server.  The \gls{mom} server talks to otter applications and integrates with the municipality. The \gls{ui} is accessed by a browser. 
    \\ \hline     
     
    Hämäläinen, Mervi
    & A Framework for a Smart City Design: Digital Transformation in the Helsinki Smart City
    & "Shed light on the elements that are relevant for robust digital transformation" \cite[p.~65]{hamalainen2020framework} by presenting a design framework
    & \begin{itemize}[leftmargin=0.3cm]
        \item Context: Helsinki Smart City
        \item Categories: Smart Cities and Design framework
    \end{itemize} 
    & Evaluation framework: 11 values that have values from 0 to 3. The 11 include four dimensions; Smart city strategy, Technology - Digital technologies, Governance - orchestration and Stakeholders, and 7 sub-values; capabilities, data, technology experimentation, security and privacy, vertical and horizontal scope, funding and metrics, and stakeholder values. 

    \\ \hline
     
    Abraham, Ralf - Aier, Stephan - Winter, Robert
    & Crossing the line: overcoming knowledge boundaries in enterprise transformation 
    &  Understanding properties of \gls{ea} that allow shared understanding during enterprise transformations
    & \begin{itemize}[leftmargin=0.3cm]
        \item Context: Enterprise transformation research
        \item Categories: \gls{ea}, Knowledge boundaries and Enterprise transformation
    \end{itemize} 
    & See \ref{fig:knowledge-boundary-properties}
    \\ \hline
     
    Mamkaitis, Aleksas - Bezbradica, Marija -Helfert, Markus 
    & Urban Enterprise: a review of Smart City frameworks from an Enterprise Architecture perspective 
    & Understand EA in smart cities 
    & \begin{itemize}[leftmargin=0.3cm]
        \item Context: Smart city research
        \item Categories: Smart Cities, \gls{ea} and \gls{togaf}
    \end{itemize}  
    & Suggests using \gls{adm}
    \\ \hline
     
    Pourzolfaghar, Zohreh - Bezbradica, Marija - Helfert, Markus 
    & Types of IT architectures in smart cities–a review from a business model and enterprise architecture perspective 
    & Evaluate architectures based on business perspective 
    & \begin{itemize}[leftmargin=0.3cm]
        \item Context: \gls{ea} business Layer research
        \item Categories: \gls{ea}, Business perspective and Smart city
    \end{itemize}  
    & Suggests using \gls{adm}
    \\ \hline
     
    Varaee, Touraj  - Habibi, Jafar - Mohaghar, Ali 
    & Presenting an Approach for Conducting Knowledge Architecture within Large-Scale Organizations 
    & Finding a valid methodology and framework for \gls{ka} within large scale organisations.
    & \begin{itemize}[leftmargin=0.3cm]
        \item Context: Large scale organisations research
        \item Categories: \gls{ea}, Knowledge and \gls{ka}
    \end{itemize}
    & \gls{ka} framework: Rectangular cuboid  (7 by 6 by 6) based on zachman
    \\ \hline
     
    L. LouwI, - H.E. EssmannII - N.D. du PreezI - C.S.L. Schutte 
    & Architecting the enterprise towards enhanced innovation capability 
    & Proposing a \gls{eaf} to support innovation
    & \begin{itemize}[leftmargin=0.3cm]
        \item Context: Enterprise research
        \item Categories: \gls{ea} and Innovation capabilities 
    \end{itemize}
    & \gls{eaf}: consisiting of strateguc intent, value chain and process, information, human resources, physical assets, organisational, performance, financial and governance architecture. It is viewed as influenced by suppliers partners customers and external influences.
    \\ \hline
     
    Närman, Pia -  Johnson, Pontus - Gingnell, Liv 
    & Using enterprise architecture to analyse how organisational structure impact motivationand learning 
    & Proposing an evaluation framework of motivation and learning based on \gls{ea} 
    & \begin{itemize}[leftmargin=0.3cm]
        \item Context: Organisational structures research
        \item Categories: \gls{ea}, motivation and learning
    \end{itemize}  
    & Evaluation model: based on \gls{uml} and \gls{ocl}
    \\ \hline
     
    \caption{Related work relevant for this thesis}
    \label{tab:related-works}
\end{longtable}

\setlength\LTleft{0cm}
\setlength\LTright{0cm}

\iffalse % Summaries if those are relevant
%A Conceptual Enterprise Architecture Framework for Smart Cities - A Survey Based Approach
& Aims to find important properties of smart cities and propose an appropriate \gls{ict} architecture. It also aims to understand the current business aspects of the \gls{it} support infrastructure. A questionnaire was used and found organisational structure, business processes, information systems and infrastructure to be most important. Suggests a generic infrastructure with a focus on interoperability. 

%A Framework for a Smart City Design: Digital Transformation in the Helsinki Smart City
& Presents smart city projects as digital transformation that changes the capabilities and organisational structure of the organisation (city) and need consideration of long term effects. The projects should allow the city to continuously evolve with long term goals. It stresses that it is a complex system and advocates for open data where possible. It found that quadruple helix collaboration had fostered technology acceptance in the city. An evaluation framework is presented
\fi


Some of the related literature used in this paper is summarised in table \ref{tab:related-works}. 

\section{summary}
There is substantial research on smart city and how it relates to \gls{ea}, but little is documented on how cities learn from \gls{ea}.