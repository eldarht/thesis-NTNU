\setlength\LTleft{-3.3cm}
\setlength\LTright{+3.3cm}
\begin{longtable}{
    |p{\textheight/7}|p{\textheight/8}|p{\textheight/10}|p{\textheight/8}|
     p{\textheight/3}|
}
    \hline
    Authors & article & Purpose & context and categorisation & Model \\ \hline
        
    Kakarontzas, George - Anthopoulos, Leonidas - Chatzakou, Despoina -Vakali, Athena
    & A Conceptual Enterprise Architecture Framework for Smart Cities - A Survey Based Approach 
    & Propose generic \gls{ict} architecture 
    & \begin{itemize}[leftmargin=0.3cm]
        \item Context: EADIC - (Developing an Enterprise Architecture for Digital Cities)
        \item Categories: \gls{ict} architecture and Smart Cities
    \end{itemize} 
    & ICT architecture: host organisation of an application has a \gls{ui} \gls{mvc} layer with synchronous \gls{api} calls to Business logic layer that communicates with local data storage and \gls{mom} server.  The \gls{mom} server talks to otter applications and integrates with the municipality. The \gls{ui} is accessed by a browser. 
    \\ \hline     
     
    Hämäläinen, Mervi
    & A Framework for a Smart City Design: Digital Transformation in the Helsinki Smart City
    & "Shed light on the elements that are relevant for robust digital transformation" \cite[p.~65]{hamalainen2020framework} by presenting a design framework
    & \begin{itemize}[leftmargin=0.3cm]
        \item Context: Helsinki Smart City
        \item Categories: Smart Cities and Design framework
    \end{itemize} 
    & Evaluation framework: 11 values that have values from 0 to 3. The 11 include four dimensions; Smart city strategy, Technology - Digital technologies, Governance - orchestration and Stakeholders, and 7 sub-values; capabilities, data, technology experimentation, security and privacy, vertical and horizontal scope, funding and metrics, and stakeholder values. 

    \\ \hline
     
    Abraham, Ralf - Aier, Stephan - Winter, Robert
    & Crossing the line: overcoming knowledge boundaries in enterprise transformation 
    &  Understanding properties of \gls{ea} that allow shared understanding during enterprise transformations
    & \begin{itemize}[leftmargin=0.3cm]
        \item Context: Enterprise transformation research
        \item Categories: \gls{ea}, Knowledge boundaries and Enterprise transformation
    \end{itemize} 
    & See \ref{fig:knowledge-boundary-properties}
    \\ \hline
     
    Mamkaitis, Aleksas - Bezbradica, Marija -Helfert, Markus 
    & Urban Enterprise: a review of Smart City frameworks from an Enterprise Architecture perspective 
    & Understand EA in smart cities 
    & \begin{itemize}[leftmargin=0.3cm]
        \item Context: Smart city research
        \item Categories: Smart Cities, \gls{ea} and \gls{togaf}
    \end{itemize}  
    & Suggests using \gls{adm}
    \\ \hline
     
    Pourzolfaghar, Zohreh - Bezbradica, Marija - Helfert, Markus 
    & Types of IT architectures in smart cities–a review from a business model and enterprise architecture perspective 
    & Evaluate architectures based on business perspective 
    & \begin{itemize}[leftmargin=0.3cm]
        \item Context: \gls{ea} business Layer research
        \item Categories: \gls{ea}, Business perspective and Smart city
    \end{itemize}  
    & Suggests using \gls{adm}
    \\ \hline
     
    Varaee, Touraj  - Habibi, Jafar - Mohaghar, Ali 
    & Presenting an Approach for Conducting Knowledge Architecture within Large-Scale Organizations 
    & Finding a valid methodology and framework for \gls{ka} within large scale organisations.
    & \begin{itemize}[leftmargin=0.3cm]
        \item Context: Large scale organisations research
        \item Categories: \gls{ea}, Knowledge and \gls{ka}
    \end{itemize}
    & \gls{ka} framework: Rectangular cuboid  (7 by 6 by 6) based on zachman
    \\ \hline
     
    L. LouwI, - H.E. EssmannII - N.D. du PreezI - C.S.L. Schutte 
    & Architecting the enterprise towards enhanced innovation capability 
    & Proposing a \gls{eaf} to support innovation
    & \begin{itemize}[leftmargin=0.3cm]
        \item Context: Enterprise research
        \item Categories: \gls{ea} and Innovation capabilities 
    \end{itemize}
    & \gls{eaf}: consisiting of strateguc intent, value chain and process, information, human resources, physical assets, organisational, performance, financial and governance architecture. It is viewed as influenced by suppliers partners customers and external influences.
    \\ \hline
     
    Närman, Pia -  Johnson, Pontus - Gingnell, Liv 
    & Using enterprise architecture to analyse how organisational structure impact motivationand learning 
    & Proposing an evaluation framework of motivation and learning based on \gls{ea} 
    & \begin{itemize}[leftmargin=0.3cm]
        \item Context: Organisational structures research
        \item Categories: \gls{ea}, motivation and learning
    \end{itemize}  
    & Evaluation model: based on \gls{uml} and \gls{ocl}
    \\ \hline
     
    \caption{Related work relevant for this thesis}
    \label{tab:related-works}
\end{longtable}

\setlength\LTleft{0cm}
\setlength\LTright{0cm}

\iffalse % Summaries if those are relevant
%A Conceptual Enterprise Architecture Framework for Smart Cities - A Survey Based Approach
& Aims to find important properties of smart cities and propose an appropriate \gls{ict} architecture. It also aims to understand the current business aspects of the \gls{it} support infrastructure. A questionnaire was used and found organisational structure, business processes, information systems and infrastructure to be most important. Suggests a generic infrastructure with a focus on interoperability. 

%A Framework for a Smart City Design: Digital Transformation in the Helsinki Smart City
& Presents smart city projects as digital transformation that changes the capabilities and organisational structure of the organisation (city) and need consideration of long term effects. The projects should allow the city to continuously evolve with long term goals. It stresses that it is a complex system and advocates for open data where possible. It found that quadruple helix collaboration had fostered technology acceptance in the city. An evaluation framework is presented
\fi

