\chapter{Results and discussion}
\label{chap:result}
This section describes the results of the research and discusses how it relates to earlier work. It also covers shortcomings and possible sources for errors.

\section{Findings for \textbf{RQ1:} How is EA currently being used to enhance learning in smart city projects?}
The literature review found that most \gls{ea} used in \gls{ict} city planning does not specifically consider learning to a great extent. The literature is conflicted on whether or not \gls{ea} in smart city projects should be used to allow replication or should focus on flexibility. The literature suggests either \gls{ict} architectures when advocating replication or existing \glspl{eaf} when advocating flexibility. Most of the research using existing \glspl{eaf} suggests using \gls{adm} and a few suggests using zachman. 
Although most literature does not specifically relate to learning, it does suggest that \gls{adm} and other \glspl{eaf} cover important aspects of it. In particular the business aspects from the different \glspl{eaf} are seen as important. The survey indicate that the \gls{eaf} used in +CityxChange has high relevance to the project which is considered to be a key factor in \glspl{boundary object}. 
The complexity of the +CityxChange \gls{eaf} is a significant hindrance for learning. The research in this thesis could not determine if this was a result of the \gls{eaf} itself or inherent to \gls{adm} that it builds on or \gls{ea} in general. The suggested changes in the proposed model were unable to lower complexity or show any significant improvement in learning.

\section{Findings for \textbf{RQ2:} How can cities benefit from EA documentation of working smart city solutions?}
The literature is split on whether or not replication is achievable, but it is clear that \gls{ea} is part of the solution. Literature on \glspl{boundary object} shows that \glspl{boundary object} such as \gls{ea} models can be useful for learning as long as the information is closely related to the domain of interest where the learning takes place. The proposed model suggested adding information on internal teams, knowledge flows and knowledge processes, notation to simplify visualisation and more elements specifically for smart city development. The evaluation found that the proposed model did not improve the +CityxChange \gls{eaf}. 
This thesis can not conclusively determine how the model could be improved, but it is believed that the core problems of the proposed model is the attempt at modelling knowledge at a teams level and not at the enterprise level. The literature on learning and innovation, and the model evaluation indicate that processes for learning are complex and require detailed descriptions to be useful. It is unclear if modelling learning on the enterprise level would have a significant benefit. 
Without the modelling of knowledge processes and flows, the +CityxChange \gls{eaf} is still a valid \gls{boundary object} that could be useful for smart city projects. The survey responses indicated that both \gls{ea} and the +CityxChange \gls{eaf} were seen as useful, but respondents were more positive towards \gls{ea} than the \gls{eaf}. This does not prove that the \gls{eaf} is more or less useful than \gls{adm}, zachman or any other framework, but it does indicate that the use of \gls{ea} in +CityxChange could improve. 


\section{Findings for \textbf{RQ3:} How can EA be used to enhance transfer of knowledge from lighthouse cities to follower cities?} The survey shows that the organisations involved vary greatly. This is also reflected in the literature and mentioned as a core challenge of replication of smart city projects. This thesis will therefore argue for using a flexible \gls{eaf} instead of focusing on replicating \gls{ict} architectures. 
The limited experience with \gls{ea} also creates a problem, as it can not guarantee a shared syntax between the communities. The evaluation of the proposed model indicate that it increased complexity and should therefore not be used. The +CityxChange \gls{eaf} or \gls{adm} should be used instead. If the +CityxChange could be made less complex without any significant side effects, then that would be ideal. This thesis can not determine how to do that as the proposed changes did not help.



\section{Findings for \textbf{RQ4:} What should EAF capture to enhance learning in lighthouse projects?} 

From the perspective of \glspl{boundary object}, what needs to be captured must relate specifically to what is being learnt. The survey and evaluation indicate that the \gls{ea} should give a high level view, hence the \gls{ea} should capture key factors for decision making at a high level. The context layer should be a focus, as it is seen as the motivation for the \gls{ea} and \gls{ict} should be present as it adds concreteness. 
The proposed model tried to model knowledge flow, but the evaluation found that this is not appropriate. It is still believed that knowledge flow should be a consideration for the \gls{ea} architect and management. This thesis can not determine more specifically what needs to be captured as the proposed changes were not seen as beneficial. 
