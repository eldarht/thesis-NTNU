\chapter{Model evaluation}
\label{chap:evaluation}
\section{Purpose of evaluation}
An evaluation of the proposed model was performed to find out if the proposed changes were beneficial for smart city projects and could enhanced learning. 

\section{Interview findings}
Participant 1 was positive to how the elements and development process related to the \gls{eaf}, but had concerns about the \gls{eaf} itself. The participant mentioned that the distinction between "Enterprise cooperation" layer and "Team structures" was not clear and that the motivation for adding "Team structures" was not clear either. "Team structures" and "internal processes" seemed to be ill fit for multi stakeholder systems such as +CityxChange and did not inherently seem to be related to theories found in \gls{togaf} or similar frameworks. It seems that these two layers were trying to model the system on a different level then the other layers and could cause conflicts. An example that could cause conflict were if the "Goals and \glspl{kpi}" layer contained a goal, then it would be hard to note if the goal was connected to the internal teams in the layers above or the "Enterprise cooperation" layer. It was also unclear if the context layer was sufficiently covered by the three layers the proposed model replaced it with. Participant 3 also mentioned the importance of the context layer as it relates to learning. Participant 3 mentioned that "Goals and \glspl{kpi}" seemed to be for quantitative aspects, but lacked the qualitative aspects of context.

"Team structures" was mentioned by participant 1 as not being particularly important, that the important parts for +CityxChange were responsibilities and decisions. Participant 2 also mentioned that they did not see the motivation behind "Team structures" and elaborated on how it clashed with the over aching goals of the \gls{eaf} to model the cooperation between the partners and stakeholders while allowing the individual partners to steer their own development process. Participant 2 mentioned that if "Team structure" and "Processes and internal initiatives" were added, then they were more likely to be vertical layers similarly to "Stakeholder perspective". The layers could still be relevant but seemed to be misplaced. Participant 3 viewed "Team structures" as what they would consider "Institutional aspects". Although participant 2 and 3 both had issues with the layer, they also mentioned that it was important in some cases.

Participant 3 went into more detail on "Processes and internal initiatives". It was considered to be very important for learning, but also very complicated with many factors. They expect that the \gls{ea} model would have to overlook important aspects of learning related processes. Participant 3 also mentioned that peoples cultures or backgrounds would have a significant effect on processes and learning.

In regards to the suggested development process, participant 2 was mostly positive, but found the concept of "mindset" to be easily misunderstood. The idea behind it was still good, but other terminology should be considered. The specific questions listed in the mindsets were relevant, but seemed more relevant for evaluating existing architectures or understanding the motivation behind the components included in a finished model. Participant 3 mentioned that several steps and vocabulary was overlapping. As an example, it would be difficult to separate users of the model from stakeholders. This would also be a problem when developing the \gls{ea} model as it is usually developed by reading documentation and interviewing users or stakeholders. It was not clear from the development process who to communicate with and how they affect your mindset.

Participant 2 went into more detail than participant 1 when evaluating the proposed elements. Altering the interface element as suggested was not recommended, but they requested adding variants or notations to specify more details about \glspl{api}. For instance, there might be an \gls{api} that will exist in the future, an anonymous \gls{api} or request format. The "indirect relationship" was seen as problematic, as everything would in some way be indirectly connected and the graphical parts seemed like a poor visualisation of indirect relationships. It was also recommended to change the "broken relationship" visualisation as the more intricate notation made it seem like a stronger connection rather than broken. Adding more relationship notations was still seen as a good idea. Participant 2 thought the idea behind the resource elements were good, but only relevant at too high a detail for an \gls{ea} that was meant for a higher level view. Participant 3 also mentioned that the resulting model would likely be too complicated to be useful for most people. For the physical elements participant 2 requested that the "physical location/building" should allow for nesting while the "measurement device" should be a collection of devices and should have a different notation. A thermometer was not seen as appropriate for the "measurement device". Participant 2 mentioned that the knowledge elements would probably not be relevant because they would need to model at too fine a detail.