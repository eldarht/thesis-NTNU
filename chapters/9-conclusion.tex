\chapter{Conclusion, limitations and future work}
This chapter summarises the thesis and presents lessons learned, implication of the findings and future work.

\section{Summary}
The motivation for this thesis was to better understand the role of \gls{ea} and how it relates to learning and knowledge transfer in smart city projects. The literature was reviewed and a survey was conducted with +CityxChange. This was used to inform a model based on the +CityxChange \gls{eaf} and later evaluated with the use of expert evaluation. 
The proposed model was found to not have a positive impact on learning, but does show that more research is needed to understand how \gls{ea} relates to learning. 

\section{Contribution / implications of study}
This thesis contributes to the current \gls{ea} research by identifying a need for a better understanding of how \gls{ea} is used in knowledge processes and how knowledge processes should effect \gls{ea} models.
The research has identified that the complexity and terminology used in \gls{ea} and smart city projects is a limiting factor for its usefulness and that supplementing \gls{ea} models with information on knowledge flow can increase complexity in a detrimental way. 

\section{limitations}
Time constraints limited the research in this thesis. A follow up questionnaire and iterative model development were planned, but not conducted. 
A follow up questionnaire would have allowed for more specific questions relating more closely to the research questions and objectives. The original questionnaire results indicated that the current situation had problems and allowed hypothesis to be formed, but without a follow up questionnaire these hypothesis could not be tested thoroughly. 

Iterative model development could have responded to the evaluation and allowed for alternative representation of knowledge processes and responded to the issues found. Without this its unclear if faults are due to the model or inherent to adding knowledge processes to \gls{ea}.

The limitations discussed have resulted in the findings being described more like an outline than specific criteria for \gls{ea}. This outline is however in line with the literature that advocate for flexibility in smart city related \gls{ea}.

\section{Future works}

Further research should be conducted to better understand how \gls{ea} relates to learning. 
It should look at how \gls{ea} could be used in existing knowledge sharing activities and documentation processes such as workshops, scrum meetings, pitch meetings and interviews.
alternatively it should look at how knowledge flows and processes could be represented in a helpful way for management. 
It should also look at how organisational cultures differ in lighthouse city projects and how that impacts \gls{ea}.